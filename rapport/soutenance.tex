\documentclass[french]{beamer}

\usepackage[utf8]{inputenc}
\usepackage[T1]{fontenc}
\usepackage[english,french]{babel}
\usepackage{graphicx}
\usepackage{amsmath}
\usepackage{fancyvrb}
\usepackage{listings}
\usepackage{color}
\usepackage{enumerate}
 
\definecolor{codegreen}{rgb}{0,0.6,0}
\definecolor{codegray}{rgb}{0.5,0.5,0.5}
\definecolor{codepurple}{rgb}{0.58,0,0.82}
\definecolor{backcolour}{rgb}{0.95,0.95,0.92}

\lstdefinestyle{mystyle}{
    backgroundcolor=\color{backcolour},   
    commentstyle=\color{codegreen},
    keywordstyle=\color{magenta},
    numberstyle=\tiny\color{codegray},
    stringstyle=\color{codepurple},
    basicstyle=\footnotesize,
    breakatwhitespace=false,         
    breaklines=true,                 
    captionpos=b,                    
    keepspaces=true,                 
    numbers=left,                    
    numbersep=5pt,                  
    showspaces=false,                
    showstringspaces=false,
    showtabs=false,                  
    tabsize=2
}

\lstset{language=C++,
    basicstyle=\small,
    keywordstyle=\color{blue}\small,
    stringstyle=\color{red}\small,
    commentstyle=\color{green}\small,
    morecomment=[l][\color{magenta}]{\#},
    tabsize=2
}

\definecolor{darkBlue}{RGB}{20,20,195}

\title{Projet Tétris}
\author{MAILLARD, ROYER, DILLON}
\date{13 mars 2019}


\usetheme{Madrid}
\useinnertheme{circles}
\usecolortheme{whale}


\begin{document}

\AtBeginSection[]{
	\begin{frame}{Plan}
		\tableofcontents[currentsection]
	\end{frame}
}


\begin{frame}

	\begin{columns}

		\hspace{-2.5cm}
		\begin{column}{0.2\textwidth}
			\includegraphics[scale=0.3]{img/logU.png}
	
		\end{column}
		
		\begin{column}{0.2\textwidth}
			\vspace{0.1cm}
			\includegraphics[scale=0.045]{img/logF.jpg}

		\end{column}

	\end{columns}

	\vspace{-0.15cm}

	\center{\textsc{\Huge \color{darkBlue} Le Vouitris}}

	\vspace{0.2cm}

	\includegraphics[scale=0.15]{img/vouitris.png}

	\center{\textsc{Cynthia MAILLARD, Félix ROYER, Alexandre DILLON}}
	\vspace{-0.20cm}
	
	\center{\textsc{\large 13 mars 2019}}

	\vspace{0.20cm}

	\textsc{\underline{Tuteur de projet} : Julien BERNARD}

\hspace{2cm}

\end{frame}

\begin{frame}{Plan}

	\tableofcontents

\end{frame}


\section{Adaptation du jeu d'origine}
	\subsection{Le jeu originel}

\begin{frame}{Adaptation du jeu d'origine}{Le jeu originel}

	\begin{block}{Le premier Tétris (1984) - Alekseï Pajitnov}
		\begin{itemize}
			\item jeu de puzzle
			\item succès mondial dans les années 1990
			\item adapté sur pratiquement toutes les consoles
			\item certaines versions mutlijoueurs
		\end{itemize}

	\end{block}

	\begin{center}
		\includegraphics[scale=0.25]{img/Tetris8.jpg}
	\end{center}
\end{frame}


\begin{frame}{Adaptation du jeu d'origine}{Le jeu originel}
	\begin{block}{La jouabilité}
		\begin{itemize}
			\item déplacement latéral et rotation des tétrominos, chute accélerée des tétromino
			\item agencer les tétromino en ligne
			\item les lignes disparaissent, rapportant du score
			\item la partie s'arrête quand les tétromino touche le plafond
		\end{itemize}

	\end{block}

	\begin{center}
		\includegraphics[scale=0.3]{img/1.png}
		\includegraphics[scale=0.4]{img/2.png}
		\includegraphics[scale=0.4]{img/3.png}
		\includegraphics[scale=0.4]{img/4.png}
		\includegraphics[scale=0.3]{img/5.png}
		\includegraphics[scale=0.3]{img/6.png}
		\includegraphics[scale=0.3]{img/7.png}
	\end{center}
\end{frame}

	\subsection{Notre adaptation}

\begin{frame}{Adaptation du jeu d'origine}{Notre adaptation}
	
	\begin{block}{Le cahier des charges}
		\begin{itemize}
			\item architecture client-serveur
			\item un joueur, un client, un ordinateur
			\item développement en C++ / Gamedev Framework / Boost::Asio
			\item développement de notre propre bibliothèque de sérialisation
		\end{itemize}
	\end{block}
	
	\begin{block}{Les objectifs}
		\begin{itemize}
			\item approfondir la programmation répartie
			\item établir un protocole réseau
			\item découvrir la sérialisation
			\item se familiariser aux bibliothèque graphiques
		\end{itemize}
	\end{block}

\end{frame}

\begin{frame}{Adaptation du jeu d'origine}{Notre adaptation}
	
	\begin{block}{Les choix d'implémentation}
		\begin{itemize}
			\item le serveur fait loi
			\item le client envoi les actions du joueur au serveur
			\item le serveur lui renvoi l'état du jeu mis à jour
			\item détruire des lignes inflige des malus à l'adversaire
			\begin{itemize}
				\item 1 ligne : pas de malus
				\item 2 lignes : empèche la rotation
				\item 3 lignes : accèlération de la chute
				\item 4 lignes : suppression de block en bas du tableau
			\end{itemize}
		\end{itemize}
	\end{block}


\end{frame}

\begin{frame}{Adaptation du jeu d'origine}{Notre adaptation}
	\begin{block}{Les contrôles}
		\begin{itemize}
			\item $\gets$ : gauche
			\item $\to$ : droite
			\item $\downarrow$ : chute rapide
			\item \emph{SPACE} : rotation
		\end{itemize}
	\end{block}

\end{frame}


\section{Modélisation du jeu}
	\subsection{Architecture réseau}

		\begin{frame}{Modélisation du jeu}{Architecture réseau}
			\begin{center}
				\includegraphics[scale=0.25]{img/archi_reseau.png}
			\end{center}
		\end{frame}

	\subsection{Protocole d'échange de messages}

		\begin{frame}{Modélisation du jeu}{Protocole d'échange de messages}
			\begin{block}{Les messages}
				\begin{enumerate}
					\item S $\to$ C : StartGame : début de la partie
					\item S $\gets$ C : TétrominoPlaced : position et rotation du tétromino placé
					\item S $\to$ C : MalusStart : signal le début d'un malus
					\item S $\to$ C : UpdateGrid : nouvelle version des grilles
					\item S $\to$ C : NewTetromino : nouveau tétromino utilisé par le joueur
					\item S $\to$ C : MalusEnd : signal la fin d'un malus
					\item S $\to$ C : GameOver : signal la fin de le partie et donne le gagnant
				\end{enumerate}
			\end{block}
		\end{frame}

\section{Mise en oeuvre}

	\subsection{Le serveur}

	\begin{frame}{Mise en oeuvre}{Le serveur}
		\begin{columns}

			\begin{column}{0.4\textwidth}
				\begin{center}
					\includegraphics[scale=0.25]{img/serveur.png}
				\end{center}
			\end{column}


			\begin{column}{0.5\textwidth}

				\begin{block}{Les données contenues dans le serveur pour chacun des joueurs}
					\begin{itemize}
						\item grille de jeu
						\item scores
						\item un vérificateur de triche
						\item un gestionnaire de malus
					\end{itemize}
				\end{block}

				\begin{center}
					\includegraphics[scale=0.25]{img/legende.png}
				\end{center}

			\end{column}

		\end{columns}
	\end{frame}




	\begin{frame}[fragile]{Mise en oeuvre}{Le serveur}
		\begin{block}{L'exploitation des messages}
			\begin{enumerate}
				\item{Reception message TetrominoPlaced}
				\item{Vérification triche}
				\item{Mise à jour de la grille}
				\item{Mise à jour du score}
				\item{Envoi de malus à l'adversaire}
				\item{Envoi grille mise à jour aux joueurs}
			\end{enumerate}
		\end{block}

	\end{frame}


	\begin{frame}{Mise en oeuvre}{Le serveur}
		\begin{block}{Le vérificateur de triche}
			\item vérifie 
		\end{block}
	\end{frame}
























	\subsection{La sérialisation}

		\begin{frame}{Sérialisation}{Motivations}
	        \begin{block}{Objectif}
	            Permettre les échanges de structures complexes entre clients et serveur via des messages standardisés.
	        \end{block}

	        \begin{block}{Les messages}
	            Deux types de messages :
	            \begin{itemize}
	                \item Du client au serveur : structure Request\_CTS
	                \item Du serveur au client : structure Request\_STC
	            \end{itemize}
	        \end{block}

	        \begin{block}{Contenu des messages}
	            \begin{itemize}
	                \item Un type enum correspondant au type du message
	                \item Une union des différentes structures de message
	            \end{itemize}
	        \end{block}

	    \end{frame}

	    \begin{frame}{Sérialisation}{La structure}
	        \begin{block}{Modèles}
	            \begin{itemize}
	                \item GamedevFramework
	                \item SFML/Packet
	            \end{itemize}
	        \end{block}
	        \begin{block}{Structures de sérialisation}
	            Deux classes symétriques communes aux clients et au serveur : Serializer et Deserializer.
	        \end{block}
	    \end{frame}

		\begin{frame}[fragile]{Sérialisation}{Les types simples}
	        \begin{block}{Sérialisation templatée}
	            Utilisation du méthode privée templatée permettant de sérialiser tout type simple.

	            \begin{lstlisting}[language=C++, caption=Méthode de sérailisation de type simple\, data est notre tableau dynamique\, d la varaible de type T à sérailiser et writePos la position d'écriture du sérialiseur]
template <typename T>
void Serializer::serializeAnyType(T d){
    size_t size = sizeof(T);
    for (size_t i = 0; i < size; ++i) {
	    data.push_back(static_cast<uint8_t>
	    						(d >> 8*(size-i-1))); 
    }
    writePos += sizeof(T);
}\end{lstlisting}

	        \end{block}
	    \end{frame}

	    \begin{frame}[fragile]{Sérialisation}{Les types simples}
	        \begin{block}{Endianess}
	            Ordre sequentiel dans lequel sont ranger nos données sérialisées.
	            Ici, endianess de format Big-Endian.
	        \end{block}

	        \begin{center}
	        	\includegraphics[scale=0.25]{img/endianess.png}
	    	\end{center}
	    \end{frame}

	    \begin{frame}[fragile]{Sérialisation}{Les objets et les messages}
	        \begin{block}{Sérialisation d'objet}
	            Sérialisation des attributs de l'objet que l'on souhaite communiquer.
	        \end{block}

	        \begin{block}{Sérialisation de message}
	            Sérialisation du type de message.
	            \newline
	            Sérialisation des élémenets de la structure du message.
	        \end{block}
	    \end{frame}



	\subsection{Les structures de données communes}

			\begin{frame}{Les structures de données communes}{Tetromino}	

				\begin{columns}
					\begin{column}{0.45\textwidth}
						\begin{block}{La classe Tetromino}
							\begin{itemize}
								\item type
								\item position abscisse/ordonnée
								\item rotation
								\item matrices<2,4> des tétromino
								\item getCases()
								\item rotate()
							\end{itemize}
						\end{block}
					\end{column}
					\begin{column}{0.35\textwidth}

					\end{column}
				\end{columns}
				
			\end{frame}


			\begin{frame}{Les structures de données communes}{Grid}	

				\begin{block}{Les vérification des intéractions}
					\begin{itemize}
						\item rotatePossible()
						\item rightPossible()
						\item leftPossible()
						\item downPossible()
					\end{itemize}
				\end{block}

				\begin{block}{Les vérification de l'état du jeu}
					\begin{itemize}
						\item suppression ligne
						\item grille complète
						\item ajout de tétromino à la grille
					\end{itemize}
				\end{block}
				
			\end{frame}

		



	\subsection{Le client graphique}

		\begin{frame}{Mise en oeuvre}{Le client graphique}
			\begin{columns}
				\begin{column}{0.35\textwidth}
					\begin{block}{Le rôle du client}
						\begin{itemize}
							\item afficher le jeu
							\item recevoir les messages du serveur
							\item gérer les interactions du joueur
						\end{itemize}
					\end{block}
				\end{column}

				\begin{column}{0.45\textwidth}
					\includegraphics[scale=0.175]{img/vouitris.png}
				\end{column}

			\end{columns}
		\end{frame}

		\begin{frame}{Mise en oeuvre}{Le client graphique}
			\begin{block}{La zone de jeu}
				\begin{center}
					\includegraphics[scale=0.2]{img/grid.png}
				\end{center}
			\end{block}
		\end{frame}


		\begin{frame}[fragile]{Mise en oeuvre}{Le client graphique}
			\begin{block}{La boucle de jeu}				
				\begin{verbatim}
					TANT QUE enPartie :
					    SI message dans File :
					        interpreteMessage()

					    SI actionJoueur :
					        SI actionPossible :
					            faireAction()

					    SI tetromino posé : 
					        envoiMessage(TetrominoPlaced)

					    MettreAJourFenetre

					FIN TANT QUE
				\end{verbatim}

			\end{block}
		\end{frame}

		\begin{frame}[fragile]{Mise en oeuvre}{Le client graphique}
			\begin{block}{L'interprétation des messages}				
			\begin{verbatim}
				SI GameStart : 
				    lancerHorloge()

				SI NewTetromino :
				    currentTetro = nextTetro; nextTetro = newTetro

				SI UpdateGrid :
				    miseAJourGrid()

				SI Malus :
				    appliqueMalus(nbLigne)

				SI GameOver : 
				    messageFin()
				\end{verbatim}

			\end{block}
		\end{frame}

\section*{Conclusion}

		\begin{frame}{Conclusion}
			\begin{block}{Un projet complet...}
				\begin{itemize}
					\item jeu fonctionnel
					\item cahier des charges respecté
				\end{itemize}
			\end{block}

			\begin{block}{... avec des améliorations possibles}
				\begin{itemize}
					\item mode de jeu avec plus de joueurs
					\item système anti-triche plus performant
				\end{itemize}
			\end{block}
		\end{frame}

		\begin{frame}{Conclusion}
			\begin{center}
				\Large{Avez-vous des questions ?}
			\end{center}
		\end{frame}




	
\end{document}